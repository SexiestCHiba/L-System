\chapter{L-Système}

\section{Principe et fonctionnement}

\subsection{Qu'est-ce que le le L-Système ?}
Le L-système \footnote{Le système de Lindebmayer}, inventé en 1968 par un biologiste hongrois du nom de Aristid Lindenmayer, est un système de réécriture \footnote{Modèle de calcul transformant des objets syntaxiques comme des mots, des termes ou encore des graphes en appliquant des règles données.} utilisé pour la modélisation de processus de developpement et de prolifération de bactéries ou de plantes.
\subsection{Comment fonctionne-t-il ?}
Ce système de réécriture fonctionne par le biais de plusieurs spécificités :
\begin{itemize}
    \item Un alphabet : celui-ci représente l'ensemble des variables utilisées pour former des mots dans le L-système.
    \item  
\end{itemize}

\section{Exemple d'utilisation}