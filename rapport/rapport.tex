\documentclass[12pt]{report}

\usepackage[utf8]{inputenc}
\usepackage[T1]{fontenc}
\usepackage[french]{babel}
\usepackage{hyperref}
\usepackage{graphicx}
\title{CONCEPTION LOGICIELLE\\L-SYSTEME}
\author{Antonin \bsc{Boyon}\\
Thomas \bsc{Lalong}\\ 
Quentin \bsc{Legot}\\
Arthur \bsc{Page}}
\date{\today}

\begin{document}

\maketitle
\thispagestyle{empty}
\setcounter{page}{0}

\tableofcontents
\newpage

\chapter{INTRODUCTION}
\newpage
\section{Sujet et consignes}
\paragraph{}
Ce projet a pour objectif de réaliser une application appliquant des principes de programmation orientée objet en language de programmation Java.
Nous avons eut le choix entre 6 sujets différents et, après études des propositions, notre choix s'est finalement porté sur le "Générateurs de flores vidéos-ludiques" et donc la réalisation d'un simulateur de L-système végétal produisant une image 2D et 3D de l'objet par le biais de règles de réécritures.

\paragraph{}
Pour cela nous avions quelques consignes a respecter :
\begin{itemize}
    \item Intégrer un parser de L-système.
    \item Créer un moteur de réécriture.
    \item Créer un moteur de rendu graphique.
\end{itemize}

\section{Mise en place du projet}

\chapter{L-SYSTEME}
\section{Principe et fonctionnement}

\section{Utilisation pour notre projet}

\chapter{ORGANISATION ET STRUCTURE}
\section{Organisation du projet}
\section{Structure du projet}

\chapter{ELEMENTS TECHNIQUES}
\section{Parser}
\section{Moteur de réécriture}
\section{Moteur graphique}
\section{Interface}

\chapter{EXPERIMENTATIONS ET USAGE}
\section{Tests de notre logiciel}
\section{Mesure de performance}

\chapter{CONCLUSION}
\section{Récapitulatif}
\section{Propositions d'amélioration}

\chapter{BIBLIOGRAPHIE}

\chapter{ANNEXES}

\end{document}
