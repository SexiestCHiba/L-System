\documentclass[
	headsepline=on,
	footsepline=on,
	twoside=off,
	abstract=on,
	DIV=10
]{scrreprt}

\usepackage[utf8]{inputenc}
\usepackage{graphicx}
\usepackage[english, french]{babel}
\usepackage{multirow}
\usepackage[dvipsnames]{xcolor}
\usepackage[allbordercolors=white]{hyperref}
\usepackage{mdframed}
\usepackage{pgfplotstable}
\usepackage{tikz-3dplot}
\usepackage[OT1]{fontenc}
\usepackage{lipsum}
\usepackage{amsmath}
\usepackage{lscape} % permet de faire des pages en mode paysage
\usepackage{algorithmicx}
\usepackage[noend]{algpseudocode}
\usepackage{listings}
\usepackage{enumitem}
\hyphenpenalty 10000

\definecolor{link}{HTML}{4169E1}
\usepackage[bottom=2cm,footskip=8mm]{geometry}

%%% Commandes de mise page propre au projet:
\newcommand{\button}[1]{\textit{\fbox{#1}}}
\newcommand{\classe}[1]{\textit{\textbf{#1}}}
\newmdenv[
rightline=false,
topline=false,
bottomline=false,
backgroundcolor=BurntOrange!5,
fontcolor=BrickRed,
linecolor=Red,
linewidth=1pt]{problemenv}
 
\newcommand{\problem}[1]{
\begin{problemenv}
\sffamily
#1
\end{problemenv}
}

\newmdenv[
rightline=false,
topline=false,
bottomline=false,
backgroundcolor=ForestGreen!5,
fontcolor=OliveGreen,
linecolor=Green,
linewidth=1pt]{resultenv}

\newcommand{\result}[1]{
\begin{resultenv}
\sffamily
#1
\end{resultenv}
}

\newmdenv[
rightline=false,
topline=false,
bottomline=false,
backgroundcolor=Cyan!5,
fontcolor=Blue,
linecolor=NavyBlue,
linewidth=1pt]{infoenv}

\newcommand{\info}[1]{
\begin{infoenv}
\sffamily
#1
\end{infoenv}
}

\newcommand{\img}[1]{
\begin{figure}[H]
	\centering
	\includegraphics[width=0.8\textwidth]{#1}	
\end{figure}
}

\newcommand{\imgwlegend}[2]{
	\begin{figure}[H]
		\centering
		\includegraphics[width=0.8\textwidth]{#1}	
		\caption{#2}
	\end{figure}
}

% Gestion d'abstracts multiples

\newenvironment{abstractpage}
{\cleardoublepage\vspace*{\fill}\thispagestyle{empty}}
{\vfill\cleardoublepage}

\renewenvironment{abstract}[1]
{\bigskip\selectlanguage{#1}%
	\begin{center}\bfseries\abstractname\end{center}}
{\par\bigskip}

% Gestion des keywords

\newcommand{\keywords}{\sffamily\textit{Keywords : }\bfseries}

%Page style

\pagestyle{headings}
\pagenumbering{arabic}


%Title page

\titlehead{
	\includegraphics[width=0.25\textwidth]{pics/LOGO-UNICAEN_V-2.1-N.png}
	\hfill
	%\includegraphics[width=0.25\textwidth]{pics/}
}
\subject{
	\small
	Université de Caen Normandie\\
	UFR des Sciences\\
	Département Informatique\\
	\hfill\\
	2ème année de licence d'informatique
}
\title{
	\hrulefill
	%\hrulefill
	\vfill\\
	\Huge \bfseries \\L-Systeme
}
\subtitle{
	Conception logicielle\\
	\hfill
	\\
	\hrulefill
	\hfill\\
	{\normalfont Rapport de projet}
}
\author{
	\small
	\hfill\\
	Antonin \bsc{Boyon}\\
    Thomas \bsc{Lalong}\\ 
    Quentin \bsc{Legot}\\
    Arthur \bsc{Page}
}
\date{}

\newcommand{\placeholderwarning}{
\problem{CECI EST UN PLACEHOLDER. À REMPLACER AVEC LES DONNÉES INDIQUÉES.}
}

\makeglossary
%redaction guide -> https://docs.google.com/document/d/1YfxGWD0GbRxs-OLxRxoA8Sg8OuVYTSKK8HX1ScFYlFA
\begin{document}

	
	\maketitle
	\pagenumbering{Roman}
	
	\tableofcontents
	\listoffigures
	\clearpage
	
	%\begin{abstractpage}
	%	\begin{abstract}{french}
	%		\lipsum[1]
	%	\end{abstract}
	%
	%	\begin{abstract}{english}
	%		\lipsum[1]
	%	\end{abstract}
	%	\hfill\\
	%	%\keywords{test lol boup incroyable rassuré}
	%\end{abstractpage}

	
	
	
	\pagenumbering{arabic}
    \info{Placeholder info}
    \problem{Placeholder problem}
    \result{Placeholder result}
    
    \chapter{Introduction}

\section{Sujet et consignes}
Ce projet a pour objectif de réaliser une application appliquant des principes de programmation orientée objet en langage de programmation Java. Nous avons eu le choix entre 6 sujets différents et, après études des propositions, notre choix s’est finalement porté sur le "Générateurs de flores vidéos-ludiques". Il consiste en la réalisation d’un simulateur de L-système végétal produisant une image 2D et 3D de l’objet par le biais de règles de réécritures.

\info{Pour cela nous avions quelques consignes à respecter :
    \begin{itemize}
        \item Intégrer un parser de L-système.
        \item Créer un moteur de réécriture.
        \item Créer un moteur de rendu graphique.
    \end{itemize}}

Après lecture des consignes, nous avons pu entamer nos recherches.

\section{Mise en place du projet}
\label{sec:MEPprog}
Nos recherches se sont premièrement portées sur le L-Système (principalement sur Wikipedia\footnote{\href{https://en.wikipedia.org/wiki/L-system}{https://en.wikipedia.org/wiki/L-system}}) pour comprendre son fonctionnement nous donnant des informations sur comment construire notre parser et notre moteur de réécriture. Nous nous sommes ensuite renseigné sur les différents moteurs de rendu graphique que nous pouvions utiliser et notre choix s'est finalement porté sur JOGL (Java Open Graphics Library \footnote{\href{https://jogamp.org/jogl/www/}{https://jogamp.org/jogl/www/}}) qui était conseillé dans la liste des sujets, pouvant gérer un rendu 2D et un rendu 3D.
\\
\\
Suite à cela, nous avons réfléchi à la structure de notre code, ainsi qu'à une première ébauche sur laquelle nous pourrions nous baser pour débuter notre projet et un ordre de priorités ; certaines parties étant nécessaires pour que d'autres fonctionnent ou puissent être amorcées (comme le parser, les bases du système de réécriture ou encore les différents moteurs de rendu).
\\
Puis, pour terminer notre mise en place, nous avons décidé que nous rajouterions une interface ainsi qu'une fenêtre d'aide a notre futur logiciel dans le but de faciliter son utilisation.



    
	\chapter{L-Système}

\section{Principe et fonctionnement}

\subsection{Qu'est-ce que le L-Système ?}
Le L-Système \footnote{Le système de Lindebmayer}, inventé en 1968 par un biologiste hongrois du nom de Aristid Lindenmayer, est un système de réécriture \footnote{Modèle de calcul transformant des objets syntaxiques comme des mots, des termes ou encore des graphes en appliquant des règles données.} utilisé pour la modélisation de processus de développement et de prolifération de bactéries ou de plantes.

\subsection{Comment fonctionne-t-il ?}
Ce système de réécriture fonctionne par le biais de plusieurs spécificités :
\begin{itemize}
    \item Un alphabet : celui-ci est l'ensemble des variables et des constantes utilisées.
    \item Un axiome : il représente le point de départ, l'état initial du système.
    \item Des règles de réécriture : elles définissent les règles de développement du L-Système en utilisant l'alphabet donné dans le but de créer un mot.
\end{itemize}
En additionnant tous ces aspects, nous obtenons alors notre L-Système, commençant par l'axiome étant la base, puis, créant au fur et à mesure un mot grâce aux règles données (Dans la limite du nombre d'itérations imposés \footnote{Le nombre d'itérations ou nombre de générations correspond au nombre de réécritures de l'axiome pour obtenir le mot final}), tout ceci étant possible grâce à l'alphabet qui les composent.
Ce mot passera ensuite par un moteur graphique dans le but d'être modélisé.

\section{Notre L-Système}

\subsection{Alphabet}
\label{sec:Alphabet}
Notre alphabet est composés de plusieurs règles et constantes :\\
\begin{itemize}
    \item X permet de dessiner une branche et Y de ne rien dessiner, ils permettent avec certaines règles de contrôler l'évolution de notre L-Système.
    \item Il est possible de modifier l'angle d'une branche en utilisant par exemple les +, -, -35, +64y, qui donnera respectivement une orientation de 25° et -25° sur l'axe de rotation x, une rotation de -35° sur l'axe X et une orientation de 64° sur l'axe de rotation y; il n'est pas possible de modifier l'orientation de l'axe de rotation Z.
\result{Exemple : +XYX appliquera une rotation de 25° a X et aucune à Y et au 2ème X}
\item Enfin il est possible d'utiliser les crochets [] pour contrôler l'évolution et obtenir des branches à vos arbres. Ces crochets vont conserver l'état, c’est-à-dire qu'une rotation appliquée aux crochets s'appliquera a tous les éléments étant à l'intérieur des crochets. Il est possible d'imbriquer des crochets.\\
\result{Exemple : +[XYX] appliquera une rotation de 25° à XYX.}
\end{itemize}

\subsection{Axiome, règles de réécritures et nombre d'itérations}
Pour l'axiome, les règles de réécritures et le nombre d'itérations, ils seront définis par l'utilisateur dans les zones de textes de l'interface prévues a cet effet. 
Un bouton "Aide" est présent sur cette même interface aidant à comprendre et mettre en place le L-Système.

\begin{figure}[h!]
    \centering
    \includegraphics[width=0.8\linewidth]{pics/aideGUI.png}
    \caption{Fenêtre d'aide}
    \label{fig:help_frame}
\end{figure}

	\chapter{Organisation et structure}

\section{Organisation du sujet}
L'organisation du projet a été soumit a un ordre de priorité car, comme cité dans "Mise en place du projet" (voir section \ref{sec:MEPprog}), certaines parties étaient nécessaires pour que d'autres fonctionnent ou puissent être amorcées :
\begin{itemize}
    \item En premier lieu nous fallait créer l'alphabet de notre L-Système (voir section \ref{sec:Alphabet}).
    \item Mise en place de méthode Parser\#isCorrect() pour vérifier certains que la syntaxe du l-système est correcte.
    \item Mise en place d'un moteur de réécriture.
    \item Mis en place d'un parser qui transforme le mot obtenu par le moteur de réécriture en une structure de données plus facilement lisible afin d'être afficher par le moteur graphique.
    \item affichage du L-Système dans le moteur graphique
\end{itemize}

\section{Structure du projet}
\begin{itemize}
    \item engine
    \begin{itemize}
        \item Rewrite: Moteur de réécriture
        \item Element, ElementProperties et Parser: voir section \ref{sec:parser}
    \end{itemize}
    \item screen
    \begin{itemize}
        \item gl3d: Tout les objets relatifs a l'affichage 3d du L-Systeme, voir la section \ref{sec:interface3d}
        \item main: Tout les objets relatifs au menu, voir la section \ref{sec:menu}
    \end{itemize}
    \item utils: contient l'objet Pair qui est essentiel au fonctionnement du projet
\end{itemize}

\begin{figure}[h!]
    \centering
    \includegraphics[width=0.7\linewidth]{pics/diagram.png}
    \caption{Diagramme de classe de notre projet}
    \label{fig:class_diagram}
\end{figure}


	\chapter{Elements techniques}

\section{Parser}\label{sec:parser}

\section{Moteur de réécriture}

Le moteur de réécriture est chargé d'appliquer les règles de réécritures itérativement à partir de l'axiome pour obtenir le mot final.
Nous avons d'un coté l'axiome qui est le mot de départ et une liste de règle qui contient 2 données: le mot a remplacer et son remplaçant.
Le mot est réécrit le mot en 2 temps:\\
\begin{itemize}
    \item On change toutes les occurrences du mot à remplacer par \$\{identifiant de la règle\}
    \item On remplace toutes les occurrences de \$\{identifiant de la règle\} apr la règle associé
\end{itemize}
On répète cette opération n fois, pour obtenir le mot final.

\section{Moteur graphique}\label{src:interface3d}

\section{Interface principale}\label{sec:menu}

\subsection{Composition de l'interface}

\paragraph{L'interface}
 utilisateur de notre logiciel à été conçue grâce à la bibliothèque \textit{Swing} de Java. Elle se compose de trois classes, une contenant la fenêtre principale \classe{MainFrame}, un autre permettant de créer des onglets \classe{Tab} et une troisième classe gérant les événements \classe{Listener}.
 
\subsection{Classes de l'interface}

\subsubsection{MainFrame}

\paragraph{La classe \classe{MainFrame}} est une classe héritant de la classe JFrame de Swing. Elle permet de créer une fenêtre de base, de taille prédéfinie dans laquelle peuvent être placés des composants graphiques. Elle comprend aussi un bouton de fermeture qui, une fois cliqué, permet l'arrêt du programme.\\
Elle comporte ainsi une instance de la classe JTabbedPane \label{jtpane}, un conteneur graphique donc le but est de disposer ses composants sous la forme d'onglets.

\subsubsection{Tab}

\paragraph{La classe \classe{Tab} } est une classe héritant de la classe JPanel de Swing. JPanel est un composant de base dans lequel il est possible d'ajouter d'autres composants graphiques. Les intances de Tab crées sont ensuites ajoutées par la classe MainFrame à son composant de la classe JTabbedPane \ref{jtpane}.

\subsubsection{Listener}

\paragraph{La classe \classe{Listener}} est une classe implémentant certaines classes Listener de Swing (\classe{ActionListener, KeyListener et MouseWheelListener}). Elle permet de capter toutes les actions effectuées par l'utilisateur et d'appeler les méthodes correspondantes des classes de l'interface. Elle permet ainsi de créer de nouveaux onglets (Nouvelles instances de Tab) mais aussi d'en fermer ou bien encore de lancer la génération du modèle.
\section{Pair ou un tuple a 2 entrées en java}
	\chapter{Experimentations et Usages}

\section{Manuel d'utilisation}

\subsection{Préambule}
Notre application a été développé et pensé pour les versions de java supérieures ou égales à la version 8u281.
L'application fonctionne sur Linux avec une interface tournant sur les moteurs graphiques Xorg et Wayland et sur Windows 10.
Notez que pour linux, notre programme ne fonctionne que pour openjdk 8

Les archives jar de Jogl doivent se trouver dans le dossier lib selon le modèle ci-dessous (image)

\problem{Nous ne pouvons pas vous garantir si l'application fonctionne sur Mac OS X, aucun des membres de notre n'en possède un.}

\subsection{Lancement de l'application}

\info{Vous devez ouvrir un terminal à l'emplacement du dossier contenu le projet}

Pour lancer l'application, exécutez la commande \button{ant run}\\

Si vous souhaitez seulement compiler les fichiers sources dans le répertoire \textbf{bin/}, executez la commande: \button{ant compile}\\

Pour générer une archive jar dans le répertoire \textbf{build/}, executez la commande: \button{ant packaging}\\

Pour générer la javadoc dans le dossier \textbf{doc/}, executez la commande: \button{ant javadoc}\\
\info{Ouvrez ensuite le fichier \textbf{doc/index.html} ou \textbf{doc/overview-summary.html} dans un navigateur.}

Pour effectuer les tests, executez la commande: \button{ant tests}\\
\info{Un fichier \textbf{result.txt} sera générée affichant les résultats ainsi qu'une copie de la sortie standard.}

\subsection{Utilisation de l'interface utilisateur}

\paragraph{Une fois l'application lancée,} une fenêtre s'affiche (figure \ref{mainframe}). Elle contient une barre de navigation grâce à laquelle vous pouvez ouvrir soit une nouvelle génération, soit une fenêtre d'aide, ainsi qu'un onglet de génération.
\begin{figure}[h!]
    \centering
    \includegraphics[scale=0.5]{pics/MainFrameGUI.PNG}
    \caption{Fenêtre principale}
    \label{mainframe}
\end{figure}
Il ne vous reste ensuite plus qu'à renseigner votre axiome, ainsi que vos règles et de cliquer sur le bouton \button{Générer en 3D}. Le bouton \button{Close} permet de fermer l'onglet de génération et le bouton \button{Clear} de supprimer votre axiome et vos règles précédemment écrites. Grâce au compteur à droite, vous êtes en mesure de définir le nombre d'itérations de votre génération.

\info{Vous pouvez ouvrir de nouveaux onglets de génération grâce au bouton \button{Nouvelle génération} mais sachez qu'un maximum de trois fenêtres est accepté}

\subsection{Navigation dans l'interface graphique en 3D}
\label{sec:nav_3d}

Pour naviguer dans l'espace 3D, vous pouvez utiliser votre clavier ainsi que votre souris. \info{La souris n'est pas essentielle, elle permet de se déplacer facilement dans l'environnement mais un clavier suffit amplement}

\paragraph{Liste des commandes au clavier : }
\begin{itemize}
    \item \textbf{Z} $\xrightarrow{} Avancer$
    \item \textbf{S} $\xrightarrow{} Reculer$
    \item \textbf{Q} $\xrightarrow{} Aller \ \grave{a} \ gauche$
    \item \textbf{D} $\xrightarrow{} Aller \ \grave{a} \ droite$
    \item \textbf{A} $\xrightarrow{} Tourner \ la \ cam\Acute{e}ra \ \grave{a} \ gauche$
    \item \textbf{E} $\xrightarrow{} Tourner \ la \ cam\Acute{e}ra \ \grave{a} \ droite$
    \item \textbf{W} $\xrightarrow{} Prendre \ de \ la \ hauteur$
    \item \textbf{X} $\xrightarrow{} Perde \ de \ la \ hauteur$
    \end{itemize}
\paragraph{Liste des commandes à la souris :}
    \begin{itemize}
    \item \textbf{Mollette Avant} $\xrightarrow{} Zommer$
    \item \textbf{Mollette Arrière} $\xrightarrow{} D\Acute{e}zoomer$
    \item \textbf{Clic Droit} $\xrightarrow{} Maintenir \ puis \ bouger \ la \ souris \ pour \ changer \ l'orientation \ de \ la \ cam\Acute{e}ra$
    
\end{itemize}

\problem{Vous ne pouvez pas utiliser 2 touches ou plus en même temps pour naviguer. Par exemple, enfoncer les touches \textbf{Z} et \textbf{D} pour aller la direction nord-est  est impossible, il vous faut tourner votre caméra dans la direction où vous voulez aller puis appuyer sur \textbf{Z}.}

Fermez la fenêtre 3D pour pouvoir générer un nouveau L-Système sans avoir à rouvrir l'application

\section{Tests de notre logiciel}

\subsection{Possibles problèmes}

Lorsque vous tentez de générer un L-Système, celui-ci est affiché en 3D en utilisant une méthode récursive, si celui-ci est trop long cela peut entraîner une erreur de type \textit{StackOverflowError}, la fenêtre 3D restera alors blanche. Fermez la puis retentez une génération avec moins d'itérations ou tentez une génération avec d'autres règles.

Si vous lancez une génération avec beaucoup d'itérations, celle-ci peut mettre un certain temps à se générer et peut provoquer une erreur \textit{OutOfMemoryError}.

\section{Mesures de performances}

Nous avons mis en place un stress-test qui réécrit et convertit en arbre le L-Système.
Ce stress-test nécessite 4GB de RAM de libre et construira un mot de quasiment 26 millions de caractères puis celui-ci sera converti en un arbre (notez qu'il est impossible de l'afficher dans le moteur graphique car cela donnera \textit{StackOverflowError}).

\begin{figure}[h!]
    \centering
    \includegraphics[scale=0.3]{pics/stresstest.png}
    \caption{Mesure de performances}
    \label{Perf}
\end{figure}

(Test de performances effectué avec un Ryzen 5 2400g et une gtx 1050ti.)

\section{Possibles améliorations}

\begin{itemize}
    \item Améliorer l'optimisation.
    \item Rendre l'interface graphique plus conviviale.
    \item Réaliser les L-Système 2D dans un vrai environnement 2D, et non dans un moteur 3D comme actuellement.
    \item Améliorer le réalisme de la modélisation 3D.
    \item Exporter les modèles d'arbres afin de les implémenter facilement dans des applications comme des jeux.
    \item Utilisation des méthodes de OpenGL 3 ou 4 au lieu de ceux de OpenGL 2 afin de profiter des dernières améliorations technologiques des cartes graphiques (notamment au niveau des performances).
    
\end{itemize}

	\input{chapters/chapitre6.tex}
	%\input{chapters/validation.tex}
	%\input{chapters/conclusion.tex}
	\cleardoublepage
	\pagebreak
	
	\pagenumbering{roman}
	\chapter{Annexes}
        \section{Remerciement}
		    Triss Jacquiot pour le modèle de rapport bien plus beau que l'original
		\addcontentsline{toc}{section}{6.2\quad{}Bibliographie}
		    \begin{thebibliography}{}
		    \bibitem{ano05}
                A. Nonymous et al.\ 2005
            \bibitem{oe04}
                A.N. Other \& S.O.M. Ebody 2004
		    \end{thebibliography}
	
	
\end{document}